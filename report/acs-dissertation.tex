%% 
%% ACS project dissertation template. 
%% 
%% Currently designed for printing two-sided, but if you prefer to 
%% print single-sided just remove ",twoside,openright" from the 
%% \documentclass[] line below. 
%%
%%
%%   SMH, May 2010. 

% TODO don't stop time spent scoring
% TODO roc_auc_score to score
% TODO convert sparse arff files to non sparse
% TODO rnd forest is bad at data thats heavily pos or neg
% TODO describe gradent problems
% TODO describe classifier scoring mechanism
% TODO optimisation: gpml's didn't work, rolled my own based on slice optimisation; explain grad based optimi, show nlml func to justify that it's simple
% TODO implement better optimiser based on gradients
% TODO comparing model evidence
% TODO look at spearmint
%
% TODO TODO start saving graphs!!

% TODO If you use purely-numeric bibliographic references, do not forget to still mention authors’ surnames

% TODO Appendices should be avoided where possible, but may be appropriate in some cases
% TODO mention ec2

% TODO Where a project has as its main aim the production of a piece of software, the project report should state clearly what test procedures were adopted and should include test output
% TODO The report should explicitly describe the starting point for the project, making clear what existing software or other resources were used

% TODO Datasets
%- Autoweka data taken from their website (how were those datasets selected?)
%- merge training and test arffs using weka cmd line
%python:
%- load using scipy.io.arff.loadarff
%- vectorise (one-of-K) using sklearn.feature_extraction.DictVectorizer

\documentclass[a4paper,12pt,twoside,openright]{report}


%%
%% EDIT THE BELOW TO CUSTOMIZE
%%

\def\authorname{Jan S\"ondermann\xspace}
\def\authorcollege{Selwyn College\xspace}
\def\authoremail{jjes2@cam.ac.uk}
% TODO Title
\def\dissertationtitle{@@@Title@@@}
% TODO word count
\def\wordcount{0}


\usepackage{epsfig,graphicx,parskip,setspace,tabularx,xspace} 

%% START OF DOCUMENT
\begin{document}


%% FRONTMATTER (TITLE PAGE, DECLARATION, ABSTRACT, ETC) 
\pagestyle{empty}
\singlespacing
\input{titlepage}
\onehalfspacing
\input{declaration}
\singlespacing
\newpage
{\Huge \bf Abstract}
\vspace{24pt} 


% TODO write abstract
%Our project falls into the field of meta-machine learning that tries to use machine learning methods to improve the result of using these methods.


\newpage
\vspace*{\fill}


\pagenumbering{roman}
\setcounter{page}{0}
\pagestyle{plain}
\tableofcontents
\listoffigures
\listoftables

\onehalfspacing

%% START OF MAIN TEXT 

\chapter{Introduction}
\pagenumbering{arabic} 
\setcounter{page}{1} 

- context
- motivation
- general overview
% TODO context (why did you do it (motivation), what was the hoped-for outcome (aims) --- as well as trying to give a brief overview of what you actually did)


% TODO It's often useful to bring forward some ``highlights'' into  this chapter (e.g.\ some particularly compelling results, or  a particularly interesting finding). 

% TODO It's also traditional to give an outline of the rest of the document, although without care this can appear formulaic  and tedious. Your call. 


\chapter{Background} 
- ML + algos
- GPs/GPML

% TODO A more extensive coverage of what's required to understand your work. In general you should assume the reader has a good undergraduate degree in computer science, but is not necessarily an expert in the particular area you've been working on. Hence this chapter may need to summarise some ``text book'' material. 



\chapter{Related Work} 

- hyper param optim
- multi armed bandits

% TODO This chapter covers relevant (and typically, recent) research which you build upon (or improve upon). There are two complementary goals for this chapter: - to show that you know and understand the state of the art; and to put your work in context

% TODO Ideally you can tackle both together by providing a critique of related work, and describing what is insufficient (and how you do better!)




\chapter{Design and Implementation} 

\section{Modelling}
% TODO different names for these two subsections
\subsection{Implementing an appropriate kernel}
\subsection{Adapting hyperparameters}

% TODO for this subsection: show plots of likelihood function
% TODO maybe include pseudocode for every step?

first we tried using the minimisation function that comes with gpml. that, however, only worked occasionally and most of the time stopped optimising after a very small number of steps when the minimum clearly hadn't been reached yet.
% TODO mention numerical instability
% TODO this algorithm is "too clever"


we then tried slice sampling, a sampling method that does not optimise the parameter but instead integrates them out and @draws samples from the posteriour?@. this worked quite well in the limited number of cases we tested.

because we wanted an optimisation technique @why? fast. ease of use@, we decided to implement an optimisation method that works similarly to slice sampling @is there any literature on this?@ this algorithm works by considering a window around the current value, updating if a smaller y was found and narrowing the window size otherwise. pseudo code for this is shown in X.

we also implemented a gradient based approach (b/c otherwise moving in a direction that's not axis aligned is slow) but that never really worked (describe nonetheless, also describe attempts to make it work

this worked fairly well but not in @more complex? noisier@ examples. describe problem with cholesky decomposition. noise param was optimised to very low value, overfitting? show diagram

After observing that restarting this algorithm with data for which it failed with a different random seed sometimes successfully optimised the parameters, we implemented a wrapper around our algorithm that reruns it multiple times, ignoring the result for cholesky errors.

this produced satisfying results but took a long time to optimise. to cut down the run time, we added a final improvement by making the algorithm terminate early if y doesn't get updated in 5 successive iterations. this happens often and makes the algorithm fast. figure X shows pseudocode for this final version of the algorithm

1. modelling
to model, two things are necessary: 1. appropriate kernel function, 2. inference
for inference, we use optimisation
- optimisation
- 1d/2d

2. decisions

\chapter{Evaluation} 
\section{Data sources}
- explain where the data came from and how it was preprocessed


- model evidence
- testing
- how well it performs


\chapter{Summary and Conclusions} 

% TODO Depending on the length of your work, and  how well you write, you may not need a summary here. You will generally want to draw some conclusions, and point to potential future work. 




\appendix
\singlespacing

\bibliographystyle{unsrt} 
%\bibliography{dissertation} 

\end{document}
